\documentclass[10pt,a4paper,notitlepage,oneside,twocolumn]{abst_jsarticle}
% notitlepage : \titlepage は独立しない
% oneside : 奇数・偶数ページは同じデザイン
% twocolumn : 2段組
% \setlength{\textwidth}{\fullwidth}
% 本文領域はページ一杯で, 傍注の幅を取らない

\usepackage[dvipdfmx]{graphicx, color}
% \usepackage{amsmath}
\usepackage{amsmath,amssymb}
\usepackage{comment}

\usepackage{url}
\usepackage{bm}
\usepackage{here}
\usepackage{algorithm}
\usepackage{algpseudocode}
\usepackage{hhline} 
\usepackage[hang,small,bf]{caption}
\usepackage[subrefformat=parens]{subcaption}
% \usepackage{tabularx}
% \usepackage[dvipdfm]{graphicx}
%\numberwithin{equation}{section}

\usepackage[hmargin=2truecm, textheight= 78zw]{geometry}
\columnsep=\dimexpr \textwidth - 50zw \relax

%%\unitlength=1pt
%%\renewcommand{\baselinestretch}{0.8}

\title{
{\bf 煙のシミュレーションにおける圧力項計算の差分スキームの比較}
}

\author{\begin{center}
{\large {\bf 23N8100018BC 須之内 俊樹}}\\
{\large {\bf 情報工学専攻 形状情報処理研究室}}\\
{\large {\bf 2023年7月}}
\end{center}}

\date{}
\pagestyle{empty}

\begin{document}

\maketitle
%%%%%%%%%%%%%%%%%%%%%%%%%%%%%%
\begin{figure*}[h]
\begin{minipage}[t]{0.5\linewidth}
  \centering
  \includegraphics[height=5cm,width=5cm]{fd_conf_73.png}
  \subcaption{前進差分を用いて計算した密度分布}
  \label{fig:fig1}
  \end{minipage}
  \begin{minipage}[t]{0.5\linewidth}
  \centering
  \includegraphics[height=5cm,width=5cm]{cd_smoke.png}
  \subcaption{中心差分を用いて計算した密度分布}
  \label{fig:fig2}
  \end{minipage}
%\end{tabular}
\caption{シミュレーションの密度分布を,ボクセルごとに密度によって色と透明度を色付けした.色が赤いほど密度が高く,青いほど密度が小さい.密度に従って透明度も変化させている.}
\label{fig}
\end{figure*}
\section{これまでの研究調査} \label{sec:intro}
コンピュータで流体の運動をシミュレーションすることは,流体と触れる工業製品の開発や,コンピュータグラフィックスによる流体の表現に役立っている.計算機を使った流体力学を特別に,数値流体力学 (CFD: Computational Fluid Dynamics) と呼び分けることも多い.CFDは実験で得ることが困難な,流れ場全体の詳細な情報を得ることができる.コンピュータグラフィックスの分野では,流体の動きを忠実に再現することよりも,それらしい流体の運動を,計算負荷を抑えて計算することが重視されている.流体の種類や運動によって条件を設けることで,よりそれらしい動きをシミュレーションすることができる.
式 \ref{eq:Navie}と式\ref{eq:uncompressed}をナビエ・ストークス方程式とよび,これは流体力学の支配方程式である.非線形二階微分方程式となっており,代数的に一般解を求める事ができない.
\begin{equation}\label{eq:Navie}
\frac{\partial}{\partial t}\bm{u} = - (\bm{u} \boldsymbol{\cdot}\nabla) \bm{u} - \frac{1}{\rho}\nabla + \nu\nabla^2\bm{u} + \bm{f}
\end{equation}
\begin{equation}\label{eq:uncompressed}
\nabla\boldsymbol{\cdot}\bm{u} = 0
\end{equation}

数値流体力学では,シミュレーションする空間や時間を離散化して近似解を求める.空間の離散化は,各辺が空間の座標軸に並行な計算格子を用いて空間を分割し,計算格子上に物理量を配置する格子法と,流体を粒子で表現し,粒子上に物理量を配置する粒子法がある.ある時刻で空間の物理量の分布を計算した後,時刻を時間の刻み幅分進めて,次の時刻の計算をすることを繰り返してシミュレーションを行う.

式 \ref{eq:Navie} の右辺の第一項を移流項,第二項を圧力項,第三項を粘性項,第四項を外力項とよぶ.移流項は非線形項であり,その他は線形項である.
格子法における移流項の計算方法として,式\ref{eq:semi-Lagrangian}で表されるStamのSemi-Lagrangian法\cite{semi-Lagrangian}がある.
\begin{equation}\label{eq:semi-Lagrangian}
\bm{u}_{t+\Delta t}(\bm{x}_t) = \bm{u}_t(\bm{x}-\bm{u}_t(\bm{x}))
\end{equation}
陽解法だと時間の刻み幅を大きくとると計算が安定しないが,この手法は陰解法であり,大きな時間の刻み幅に対応している.また線形補間で実装でき,実装が容易であるため,コンピュータグラフィックスでは広く用いられている.

圧力項の計算方法に,Chorinのprojection法\cite{projection}があり,仮の速度を$\bm{u}^*$として,以下のポアソン方程式を解く.
\begin{equation}\label{eq:colin_p}
\nabla^2 \bm{p} =  \frac{1}{\Delta t}\nabla\boldsymbol{\cdot}\bm{u}^*
\end{equation} 
ノイマン境界条件は,$\frac{\partial p}{\partial \bm{n}} = 0$とする.これにより,ディリクレ境界条件も設定される.これはシミュレーション境界の格子に配置される,境界外の方向の速度成分を$0$とすることで実装できる.ポアソン方程式は離散化によって,線形行列を用いた線形方程式に帰着できる.シミュレーション領域の分割数を$n$とすると,行列は$n \times n$の疎行列になる.その後,式\ref{eq:colin_v}を用いて実際の速度を更新する.

\begin{equation}\label{eq:colin_v}
\bm{u}_{t+\Delta t} = \bm{u}^* - \Delta t \nabla \bm{p}
\end{equation} 

\section{課題の発見,特定の状況}
CGにおける煙のシミュレーション手法に,Fedkiewらの\cite{fedkiew}がある.この手法はこれまでの研究調査で述べた計算法のほか,煙の温度による浮力,渦の力,重力などを外力に加え,煙の様子を流体の密度分布をもとにレンダリングする手法になっている.この手法はそれらしい煙のシミュレーションを実現しているが,計算負荷が大きく,粗い格子でもリアルタイムにシミュレーションすることができていない.流体シミュレーションは煙に限らず,表現する流体の特徴を利用して,計算負荷を抑える取り組みがなされている.また,流体シミュレーションの計算はGPUを用いた並列計算と相性が良く,シミュレーションの高速化を目指す取り組みもある.Ishidaらの手法\cite{GPU}は,煙の特徴を利用しつつ,GPUを用いて高速計算をする手法である.計算方法は大部分は\cite{fedkiew}と同じだが,煙は液体と比べ境界が曖昧なことを利用し,離散コサイン変換を用いて物理量を圧縮し,メモリ消費量を抑えながら,従来のGPU利用の手法よりも高速なシミュレーションの計算に成功している.しかし依然として計算負荷や,メモリ消費は大きく,計算機の性能によって計算時間は大きく変わってしまう.この手法では,従来の手法のメモリ消費を$1/8$に抑えることを目指したが,実験結果ではメモリ消費量は$1/2$にとどまっている.一度にGPUのグローバルメモリに送る物理量のメモリ量によってメモリ消費量は異なり,データの展開圧縮方法に工夫が必要である.

\section{研究方針}
GPUを利用した高解像度の高速なシミュレーション手法の改善,提案を目指し,まず\cite{fedkiew}を参考に格子法を用いた煙のシミュレーション手法を実装する.次に\cite{GPU}を参考に,離散コサイン変換を利用した展開圧縮手法を実装を通じて,GPUを利用した煙のシミュレーションの改善点を模索する.

\section{実施した実験と得られた知見}
%\begin{figure}[H]
   % \includegraphics[width=80mm]{fd_conf_73.png}
    %\caption{シミュレーションの密度分布を,ボクセルごとに密度によって色と透明度を色付けした.色が赤いほど密度が高く,青いほど密度が小さい.密度に従って透明度も変化させている.}
 %   \label{fig:fig1}
 %\end{figure}
 
 %\begin{tabular}{cc}



\cite{fedkiew}を参考に格子法を用いた煙のシミュレーション手法を実装し,解像度を$32\times32\times64$に設定したシミュレーション領域の,1つの面に気体を配置し,配置した面の中心に熱源を置いて,煙が上昇する様子をシミュレーションした.図\ref{fig:fig1}は,ParaViewを用いてシミュレーション結果を可視化したものである.この手法では物理量ごとにさまざまな差分近似を用いており,特に圧力項の差分近似について調査を行った.圧力項計算のポアソン方程式\ref{eq:colin_p}は圧力勾配の差分近似に前進差分を用いている.これによりシミュレーションを進めるにつれて対称性が崩れていると考えられる.対称性が崩れない差分近似に中心差分がある.前進差分が一次精度の誤差なのに対し,中心差分は二次精度であるため精度面で優れている.しかし中心差分は計算が安定せず,収束に時間がかかるため一般的にシミュレーションでは用いられない.実際に中心差分で実装した結果,収束計算が膨大になり,図\ref{fig:fig2}のようにシミュレーションが発散した.

\section{今後の研究計画}
シミュレーション結果をParaViewを用いたアニメーションによって確認するには,解像度$32\times32\times64$が限界であり,これ以上の解像度では快適に再生できない.解像度によって物理量の分布の様子は大きく異なるため,Houdiniなどの描画ソフトを用いて高解像度のアニメーションを作ることで結果を確認できるようにする.次に\cite{GPU}を参考に,離散コサイン変換を利用した展開圧縮手法を実装し,GPUを利用した煙のシミュレーションの改善点を模索する.
% 参考文献
\begin{thebibliography}{99}

\bibitem{fedkiew}
FEDKIW R., STAM J., JENSEN H. W.: Visual simulation of smoke. In Proceedings of the 28th Annual Conference on Computer Graphics and Interactive Techniques (2001), SIGGRAPH ’01, pp. 15-22. 

\bibitem{semi-Lagrangian}
STAM J.: Stable fluids. In Proceedings of the 26th Annual Conference on Computer Graphics and Interactive Techniques (1999), SIGGRAPH ’99, pp. 121-128. 

\bibitem{projection}
A. Chorin. A Numerical Method for Solving Incompressible Viscous Flow Problems. Journal of Computational Physics, 2:12-26, 1967.

\bibitem{GPU}
D. Ishida , R. Ando , S. Morishima.: GPU Smoke Simulation on Compressed DCT Space. EUROGRAPHICS 2019/ P. Cignoni and E. Miguel

\end{thebibliography}
\end{document}
